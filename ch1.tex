%!TEX root = thesis.tex


\pagenumbering{arabic}

\chapter{Introduction}
\label{chap:Introduction}

\section{The Theta Body of the Stable Set Problem}
This thesis studies the application of sums of squares to combinatorial optimization.
In this section, we give a description of the general area of combinatorial optimization and sums of squares in particular.
We introduce many of the concepts which will be used in the following chapters. 
In the remainder of the introduction, we give an extended abstract of each
chapter.

\subsection{Combinatorial Optimization Problems}
Combinatorial optimization is the branch of optimization problems concerning
finding the best object from a discrete set. Applications are common in 
computer science, mathematics, and operations research. For instance, the 
objective of traveling salesman problem is to find a minimum-cost tour among
a set of cities; the shortest path problem is to find a shortest-distance
path between two nodes in a graph; and the assignment problem is to assign
workers to jobs while minimizing the total wages paid. In particular, we will
consider the 
stable set problem, where we find the maximum-size independent vertex set in a
graph.

Combinatorial optimization problems often end up being NP-complete or harder.
That is, we are unlikely to ever find efficient algorithms to solve them
exactly. And in some cases where the abstract mathematical problem does have
an efficient algorithm, real-world constraints can make it intractable. For
instance, when assigning medical students to residencies, the constraint that
married couples live in the same city makes the problem NP-complete.

Therefore, much research has focused on developing methods to solve these
problems approximately. In this paper, we describe one such method based on
sums of squares.

\subsection{Sums of Squares and Optimization}
Consider for the moment unconstrained optimization problems on $\RR^n$. That 
is, we are given a function $f$ in $n$ real variables, and must find the 
global minimum value $f_{\min}$. We can rephrase this as $f_{\min} = 
\max \{r: f(x) - r \ge 0\}$. Now if we can write $f(x) - r$ as a sum of squares,
then it is certainly nonnegative. Therefore, we can define the relaxation
$f_{\textup{sos}} = \max \{r: f(x) - r \textup{ is a sum of squares}\}$.
We have that $f_{\min} \ge f_{\textup{sos}}$, so this is a lower bound on
the minimum. This idea is discussed at more length in \cite{sostools} and 
\cite{lasserre}. We now specialize to the case of combinatorial optimization.

\subsection{The Theta Body}
We would like to apply the $f_{\textup{sos}}$ construction above to
combinatorial optimization problems. We will explain how to do this through
an example.

The {\em stable set problem} is, given a graph $G = (V,E)$, to find the
maximum size of a vertex set which is completely disconnected. We can embed
this problem in $\RR^n$ as follows. Let $n = |V|$ and label the vertices by 
$1, \ldots, n$. For each stable set $S \subseteq V$, let $\chi_S$ be its 
characteristic vector: $(\chi_S)_i = 1$ if $i \in S$, and 0 if $i \notin S$.
The collection $X$ of these characteristic vectors forms a discrete point set
in $\RR^n$, and in fact it is an algebraic variety. Its ideal
$I$ is generated by $\{x_i^2 - x_i \, (1\le i \le n);\, x_ix_j\, (ij \in E) \}$.

The problem of finding the maximum size of a stable set can now be rephrased
over $\RR^n$ as finding the minimum value of the function
$f(x) = - \sum_{i=1}^n x_i$ over the set $X$ given above. To use the
$f_{\textup{sos}}$ idea, we need to define a notion of being a sum of squares
on $X$. As long as $X$ is an algebraic set, this is accomplished in 
\cite{glpt}. The method is to take a basis $B$ for $\RR[x]/I$ and use 
semidefinite programming to calculate the space of sums of squares mod $I$ by
computing in the basis $B$. One caveat is that we need to restrict the 
squares to some bounded degree $k$ to get a polynomial-time algorithm. If
we set $k=1$ we get the celebrated Lovasz theta body for the stable set
problem, which was the inspiration for \cite{glpt}. By increasing $k$ we
increase the runtime of our algorithm but potentially improve its accuracy;
this is discussed in chapter X. 

\subsection{Structure of the Thesis}
In the remainder of the introduction, we give an extended abstract of each
chapter. We will describe the problem and its motivation and applications,
and give a precise statement of results and an indication of how they were 
proved.

Of course, the papers themselves are present as the other chapters here, so
they are available to also read. However, the goal of this introduction
is to give a precise overview of results and be a guide to the rest of the
thesis.

\section{A Semidefinite Approach to the $K_\MakeLowercase{i}$ Cover Problem}
Chapter 2 is taken from a paper coauthored with Jo\~ao Gouveia and submitted to 
\emph{Operations Research Letters}. 

\subsection{Problem Description}

The {\em $K_i$-cover problem} is a generalization of the stable set problem discussed
in Chapter 1.1. A graph $G = (V,E)$ is given. For any $k$, a $k$-clique in $G$,
denoted $K_i$,
is a collection of $k$ nodes, each pair of which is connected by an edge in $E$.
We define a covering relation where an $i$-clique $H_1$ is covered by an 
$(i-1)$-clique $H_2$ if $H_1 \supseteq H_2$; i.e., if $H_2$ is a subgraph of
$H_1$. A $K_i$-cover in $G$ is a
collection $\mathcal{F}$ of $(i-1)$-cliques in $G$ such that each $i$-clique in $G$
is covered by some element of $\mathcal{F}$. The $K_i$-cover problem is to find such a
collection $\mathcal{F}$ of smallest size.
In the graph $G$ in Figure \ref{K5}, a $K_4$-cover is given by $\{012,034,023\}$.

\begin{figure}[htd]
	\centering
	\includegraphics[width=.4\textwidth,natwidth=613,natheight=584]{K5.png}
	\includegraphics[width=.4\textwidth,natwidth=613,natheight=584]{K4cover.png}
	\caption{A graph with 5 copies of $K_4$: 0123,0124,0134,0234,1234; along with a possible $K_4$-cover.}
	\label{K5}
\end{figure}

When $i=2$, this is the \emph{vertex cover problem} - to find the smallest
set $\mathcal{F}$ of vertices such that each edge is covered by a vertex in
$\mathcal{F}$. To make the
connection to the stable set problem, note that $\mathcal{F} \subseteq V$ is a vertex
cover if and only if $V \setminus \mathcal{F}$ is a stable set.
Therefore, the problems are essentially equivalent. In fact, the polytopes
defined by these problems as in section 1.1 are congruent via the transformation
$x_j \mapsto 1-x_j$ in each coordinate.

\subsection{Background}

The stable set and vertex cover problems have been studied in many contexts. 
The $K_i$-cover generalization was first studied in Conforti et al 
\cite{conforti}, wherein the associated polytope $P_i(G)$ was 
considered and several families of facets identified. 
As in Section 1.1, $P_i(G) \subseteq \mathbb{R}^N$ is the convex hull of
characteristic vectors of $K_i$-covers, where $N$ is the number of $K_{i-1}$s in
$G$. 
Conforti et al provided polynomial-time {\em separation oracles} for many of these families of facets. 
A separation oracle for a family $\mathcal{F}$ of facets is a decision procedure
which takes a point $x$ and decides whether $x$ satisfies each facet $F \in
\mathcal{F}$, or whether $x$ lies outside some facet $F \in \mathcal{F}$.
Since a typical family $\mathcal{F}$ will contain exponentially many elements, it is not possible in general to enumerate the $F \in \mathcal{F}$ and check them one by one, making such an oracle a nontrivial result.

Conforti et al left open the existence of oracles for several families of
facets, including the family associated with the {\em $K_i$-$p$-holes}. 
A graph $H$ is a $K_i$-$p$-hole if it contains $p$ copies of $K_i$ arranged in a cycle, with neighboring $K_i$ sharing a common $K_{i-1}$. 
This does not determine $H$ up to isomorphism; see Figure 2 for three nonisomorpic $K_3$-9-holes.
To understand the facet inequality associated to a $K_i$-$p$-hole, consider the graphs in Figure 2. 
To construct a $K_3$-cover of minimum size, we need to pick an edge from each $K_3$. 
We can pick four edges to eliminate eight $K_3$s, but we still need a fifth to pick up the last $K_3$. 
Therefore, $\sum_{e \subseteq H} x_e \ge 5$ is valid on the polytope $P_3(G)$ for any graph $G$ containing $H$ as a subgraph. 
The inequality for general $K_i$-$p$-holes is derived from a similar argument,
and is given by $\sum_{e \subseteq H} x_e \ge \lceil \frac{p}{2} \rceil$.
It defines a facet of $P_i(G)$ for $i \ge 3$ and odd $p$.

\begin{figure}[htd]
	\centering
	\includegraphics[width=.3\textwidth,natwidth=589,natheight=584]{9holeA.png}
	\includegraphics[width=.3\textwidth,natwidth=575,natheight=584]{9holeB.png}
	\includegraphics[width=.3\textwidth,natwidth=584,natheight=584]{9holeC.png}
	\caption{Three non-isomorphic $K_3$-9-holes.}
	\label{9holeintro}
\end{figure}

\subsection{Results}
\begin{itemize}
\item We show that the family of facets corresponding to $K_i$-$p$-holes is valid on
the theta body $\textup{TH}_{\lceil i/2\rceil }(G)$.
Therefore, for fixed $i$, we have a polynomial-time algorithm to optimize over a relaxation at least as tight as this family. 
To show that the $K_i$-$p$-hole facets are valid on $\textup{TH}_{\lceil i/2\rceil}(G)$, we choose a set of 
polynomials which are idempotent mod $I$, and whose sum is the facet-defining
inequality.
\item We consider the triangle free problem, the special case $i=3$ of the
$K_i$-free problem. We show that $P_3(K_n) \subsetneq \textup{TH}_{k}(K_n)$ for $k < n/2$.
That is, it takes at least $n/2$ steps for the theta body heirarchy to converge
to the triangle free polytope for $K_n$, and therefore this heirarchy does not give a polynomial-time algorithm for the triangle free problem.
We show this by observing that the cut polytope and triangle free polytope of $K_n$ share a facet, and apply a result of Laurent \cite{moniquestuff} that the theta heirarchy takes at least $n/2$ steps to reach this facet.
\item We show that there is an {\em integrality gap} of $1/2$ for the triangle cover's second theta relaxation. 
That is, if we optimize in the all-1 direction in the triangle cover problem, we have $\min \{1 \cdot x: x \in \textup{TH}_2(G) \} \ge \frac{1}{2} \min \{1 \cdot x: x \in P_3(G)\}$ for all $G$. 
We prove this by applying a result of Krivelevich \cite{krivelevich} on a fractional relaxation of the same problem, and proving that $\textup{TH}_2(G)$ is contained in this fractional relaxation.
\end{itemize}
\subsection{Comments}
We don't fully address the question raised in Conforti of whether there is a polynomial time separation oracle for the family $\mathcal{F}$ of $K_i$-$p$-hole facets.
We show that the facets are valid on $\textup{TH}_{\lceil i/2\rceil}(G)$, and that we can check membership of $\textup{TH}_{\lceil i/2 \rceil}$ in polynomial time. 
However, this does not give a separation oracle for $\mathcal{F}$.
Indeed, let $Q$ be the body defined as the intersection of all $F \in \mathcal{F}$.
We have $P_i(G) \subseteq \textup{TH}_{\lceil i/2\rceil}(G) \subseteq Q$, so in terms of an approximation to $P_i(G)$, the theta body is tighter, and we can optimize over it in polynomial time (for fixed $i$).
However, this doesn't allow us to optimize over exactly $Q$ in polynomial time.
In fact, for $i=2$, the stable set case, we have a similar phenomenon:
$\textup{STAB}(G) \subseteq \textup{TH}(G) \subseteq \textup{QSTAB}(G)$. Here STAB is the convex hull of stable sets in $G$ and QSTAB is the LP relaxation given by clique inequalities.
In this case it is NP-hard to optimize over either STAB or QSTAB, while TH can be optimized over in polynomial time.


Conforti et al \cite{conforti} found additional facets of $P_i(G)$, associated
to other subgraphs of $G$, for which no separation oracle is currently known. We
checked numerically and found that they did not appear to be valid on
$\textup{TH}_{\lceil i/2\rceil}(G)$, but they may be valid on higher theta bodies.




\section{Sums of Squares on the Unit Hypercube}

Extended abstract of the cube paper goes here.

\section{Flag Algebras and Sums of Squares}

Extended abstract of the flag algebras paper goes here.

\section{The Representation Theory of Matchings}

Extended abstract of the matchings paper goes here.

\section{A Note on Notation}
The following chapters were published as separate papers, and in some cases
refer to different aspects of the same or related problems. As such, the 
chapters use different notation to fit the topic at hand, so the same object
may have different names in different chapters.
