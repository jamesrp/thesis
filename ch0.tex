%!TEX root = thesis.tex

\titlepage
\newpage

\copyrightpage



\abstract{
\thispagestyle{empty}
This thesis studies percolation, a problem in probability, as well as several topics in the application of sums of squares to combinatorial optimization.

In the chapter on percolation, we bound the critical probability for bootstrap percolation on the Hamming torus, as well as the critical probability for $i$-dimensional subgraphs to percolate.
In the case $d=\theta=3$ we exhibit a framework for deriving exact results within the scaling window using Poisson approximation.

In the chapters on combinatorial optimization, we consider the $K_i$-cover problem and the max cut problem.
We show that a family of facets arising from $K_i$-$p$-holes is valid on the $i/2$ theta body.
We also prove an integrality gap of $1/2$ for the triangle free problem, and show that at least $n/2$ steps are required for the triangle free problem's theta bodies to converge in the case $G = K_n$.

We introduce a criterion for an invariant polynomial to be a sum of squares on the hypercube.
This gives a simple proof of Laurent's result that the theta body heirarchy requires at least $n/4$ steps to converge to the max cut polytope of $K_n$.
It also allows us to give the first lower bounds on degrees of denominators in Hilbert's 17th problem.

We consider the $S_n$-irreducible decomposition of the space of matchings on $K_n$ as given by Barbasch and Vogan.
We give an explicit map of the isomorphism in their result.
We also generalize their approach to matchings on hypergraphs.
%We prove some asymptotic results in bootstrap percolation, a probabilistic model of nucleation and growth.
%We then study the application of sum of squares methods to combinatorial optimization problems. 
%We analyze the gap between nonnegativity and sums of squares to understand the quality of certain approximations to the max cut problem and $K_i$-cover problem.
%We also prove a result on the representation theory of the space of matchings.
}

\acknowledgments{
\thispagestyle{empty}
I would like to thank Rekha Thomas for guiding me through my education in mathematical research, and for many interesting discussions over the years.
}

\dedication{
\thispagestyle{empty}
\begin{center}To Katharine\end{center}
}
