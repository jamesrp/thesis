%!TEX root = thesis.tex


\pagenumbering{arabic}

\chapter{Introduction}
\label{chap:Introduction}

\section{The Theta Body of the Stable Set Problem}
This thesis studies the application of sums of squares to combinatorial
optimization. Chapters 2-X each cover a different aspect of this research.
In this introduction, we give a case study in which we apply sums of squares
to the stable set problem, and introduce many of the concepts which will be 
used in the following chapters. Then we give an extended abstract of each
chapter.

\subsection{Combinatorial Optimization Problems}
Combinatorial optimization is the branch of optimization problems concerning
finding the best object from a discrete set. Applications are common in 
computer science, mathematics, and operations research. For instance, the 
objective of traveling salesman problem is to find a minimum-cost tour among
a set of cities; the shortest path problem is to find a shortest-distance
path between two nodes in a graph; and the assignment problem is to assign
workers to jobs while minimizing the total wages paid. In particular, we will
consider the 
stable set problem, where we find the maximum-size independent vertex set in a
graph.

Combinatorial optimization problems often end up being NP-complete or harder.
That is, we are unlikely to ever find efficient algorithms to solve them
exactly. And in some cases where the abstract mathematical problem does have
an efficient algorithm, real-world constraints can make it intractable. For
instance, when assigning medical students to residencies, the constraint that
married couples live in the same city makes the problem NP-complete.

\subsection{Sums of Squares and Optimization}

\subsection{The Theta Body}

\subsection{Structure of the Thesis}
In the remainder of the introduction, we give an extended abstract of each
chapter. We will describe the problem and its motivation and applications,
and give a precise statement of results and an indication of how they were 
proved.

Of course, the papers themselves are present as the other chapters here, so
they are available to also read. However, the goal of this introduction
is to give a precise overview of results and be a guide to the rest of the
thesis.

\section{A Semidefinite Approach to the $K_\MakeLowercase{i}$ Cover Problem}

\section{A Note on Notation}
The following chapters were published as separate papers, and in some cases
refer to different aspects of the same or related problems. As such, the 
chapters use different notation to fit the topic at hand, so the same object
may have different names in different chapters.
