\chapter{A Semidefinite Approach to the $K_\MakeLowercase{i}$ Cover Problem}
\label{chap:kicover}


\section{Introduction}
A common way to model a combinatorial optimization problem is as the optimization of a function over the set $S \subseteq \{0,1\}^n$ of  characteristic vectors of the objects in question. When the objective function is linear, 
we may replace $S$ by its convex hull $\hbox{conv}(S)$. The problem can be solved efficiently if we can find a small description of this polytope. Since for NP hard problems we cannot expect this, we look instead for  approximations to $\hbox{conv}(S)$. One possibility is to use semidefinite approximations, as introduced by Lov\'{a}sz \cite{lovasz} with the construction of the {\em theta body} of the stable set polytope of a graph. Another famous example is the approximation algorithm for the max cut problem due to Goemans and Williamson \cite{goemans_williamson}. In this paper we will use the semidefinite relaxations introduced by Gouveia, Parrilo and Thomas \cite{gpt} to analyze the {\em $K_i$ cover problem}.

Recall that $K_i$ denotes the complete graph, or clique, on $i$ vertices. Given a graph $G$, let $\KK_j(G)$ be the collection of cliques in $G$ of size $j$ (usually, the graph is clear from context, and we write $\KK_j$). A collection $C \subset \KK_{i-1}$ is said to be a $K_i$-cover if for each $K \in \KK_i$, there is some $H \in C$ with $H \subset K$. In this case we say that $H$ covers $K$. The $K_i$ cover problem is, given a graph $G$ and a set of weights on $\KK_{i-1}$, to compute the minimum weight $K_i$ cover. The case $i=2$ is more commonly known as the vertex cover problem, in which we seek a collection of vertices such that each edge in $G$ contains at least one vertex from the collection. However, note that the usage of ``cover'' is reversed here: the vertex cover problem is the $K_2$ cover problem, not the $K_1$ cover problem.

A closely related problem, and the setting in which we will prove our results, is the {\em $K_i$ free problem}. As before, we are given a graph and a collection of weights on $\KK_{i-1}$. But now we seek the maximum weight collection $C \subseteq \KK_{i-1}$ such that $C$ is $K_i$-free. That is, for each $K \in \KK_i$, there is some $H \in \KK_{i-1}$, with $H \subset K$ and $H \notin C$. Again, the case $i=2$ of this problem is well-known as the stable set problem: we seek a maximum weight {\em stable set} $C$, where $C$ is stable if no two of its vertices are connected by an edge.

The vertex cover and stable set problems are related in the following sense: let $G = (V,E)$ be a graph. Then a subset $C$ of vertices is a vertex cover if and only if $V \setminus C$ is a stable set. The same is true for the $K_i$ cover and $K_i$ free problems: a subset $C \subset \KK_{i-1}$ is a $K_i$-cover if and only if $\KK_{i-1} \setminus C$ is $K_i$-free. Therefore, for a given set of weights on $\KK_{i-1}$, optimal solutions to the two problems are complementary, and so solving one solves the other.

In this paper, we consider the polytope associated with the $K_i$ free problem. Let 
$P_i(G) = \hbox{conv}(\{\chi_S: S \subset \KK_{i-1}(G) \text{ and $S$ is $K_i$-free}\})$, 
the convex hull of the incidence vectors of the $K_i$ free sets. Note that $P_i(G) \subseteq [0,1]^{\KK_{i-1}(G)}$.

As the $K_i$ free problem is NP-complete (see \cite{conforti}), we cannot expect a small description of $P_i(G)$ for general graphs $G$. However, for certain classes of facets of $P_i(G)$, Conforti, Corneil, and Mahjoub \cite{conforti} show that we can solve the separation problem in polynomial time, allowing us to optimize efficiently over a relaxation of $P_i(G)$. We provide a tighter relaxation of $P_i(G)$, improving their result.

The structure of this paper is: in section 2, we outline the main algebraic machinery, {\em theta bodies}, a semidefinite relaxation hierarchy. In section 3 we show that the $K_i$-$p$-hole facets are valid on the theta bodies. Finally, in section 4 we focus on the triangle free problem. We show that in the case of $G = K_n$, the theta body relaxations cannot converge in less than $n/4$ steps. We also use a result of Krivelevich to show an integrality gap of 2 for the second theta body.

\section{Theta bodies}
Theta bodies are semidefinite approximations to the convex hull of an algebraic variety. For background, see \cite{frg} and \cite{gpt}. Here we state the necessary results for this paper without proofs.

Let $V \subseteq \RR^n$ be a finite point set. One description of the convex hull of $V$ is as the intersection of all affine half spaces containing $V$:
$$\hbox{conv}(V) = \{x \in \RR^n: f(x) \ge 0 \hbox{ for all linear $f$ such that } f|_V \ge 0\}.$$
Since it is computationally intractable to find whether $f|_V \ge 0$, we relax this condition. Let $I$ be the vanishing ideal of $V$, i.e., the set of all polynomials vanishing on $V$. Recall that $f \equiv g \mod I$ means $f - g \in I$, and implies that $f$ and $g$ agree on $V$. A function $f$ is said to be a sum of squares of degree at most $k$ mod $I$, or {\em $k$-sos mod $I$}, if there exist functions $g_j$, $j=1,\ldots,m$ with degree at most $k$, such that $f \equiv \sum_{j=1}^mg_j^2$ mod $I$. If $f$ is $k$-sos mod $I$ for any $k$, it is clear that $f|_V \ge 0$ since $g_j^2$ is visibly nonnegative on $V$. Therefore, we make the following definition of $\THETABODY_k(I)$, the $k$-th theta body of $I$:
$$\THETABODY_k(I) = \{x \in \RR^n: f(x) \ge 0 \hbox{ for all linear $f \equiv$ $k$-sos mod $I$}\}.$$
The reason why the theta bodies $\THETABODY_k(I)$ provide a computationally tractable relaxation of $\hbox{conv}(V)$ is that the membership problem for $\THETABODY_k(I)$ can be expressed as a semidefinite program, using {\em moment matrices} that are reduced mod $I$. 

For what follows, we will restrict ourselves to a special class of varieties, and suppose that our variety $V \subseteq \{0,1\}^n$ and is {\em down-closed}; i.e., if $x\le y$ componentwise, and $y \in V$, then $x \in V$. Additionally, we will always assume that $V$ contains the canonical
basis of $\RR^n$, $\{e_1, \cdots, e_n\}$, as otherwise we could restrict ourselves to a subspace. All combinatorial optimization problems of avoiding certain finite list of configurations, such as stable set, $K_i$ free, etc., have down-closed varieties. The restriction to this class is not necessary, but makes the theta body exposition simpler. In particular, the ideal of a down-closed variety has the following simple description.

\begin{lemma} \label{ideal}
Let $V$ be a down-closed subset of $\{0,1\}^n$. Then its vanishing ideal is given by
$$I=\langle x_j^2 - x_j: j = 1, \ldots, n;
x^S: S \notin V\rangle,$$ and a basis for $\RR[V] = \RR[x]/I$ is given by $B = \{x^S: S \in V\},$
where $x^S := \prod_{i \in S} x_i$ is a shorthand used throughout the paper.
\end{lemma}

Another important fact about $\THETABODY_k(I)$ in this setting (when $I$ is real radical) is that a linear inequality $f(x) \geq 0$ is valid on $\THETABODY_k(I)$ if and only if $f$ is actually $k$-sos modulo $I$.
In section 3, we will prove that certain facet-defining inequalities of $P_i(G)$ are also valid on its theta relaxations $\THETABODY_k(I)$ by presenting a sum of squares representation modulo the ideal. For now, we observe that by considering degrees, we can get a bound on which theta bodies are trivial; that is, equal to the hypercube $[0,1]^n$.

\begin{lemma} \label{lowerbound}
Let $V \subseteq \{0,1\}^n$ be down-closed, and suppose that all elements $x \notin V$ have $\sum_j x_j \ge k$. Let $I$ be its vanishing ideal. Then 
for $l < k/2$, $\THETABODY_l(I) = [0,1]^n$.
\end{lemma}
\begin{proof}
Let $f$ be linear with $f \equiv \sum_j g_j^2 \mod I$ with each $g_j$ of degree at most $l$. Then $f - \sum_j g_j^2 =: F \in I$, and $F$ has degree at most $2l$. But the basis from Lemma \ref{ideal} is a Groebner basis, and the only elements with degree $2l$ or less are $x_j^2 - x_j$, so $F \in I' := \langle x_j^2 - x_j ;j = 1, \ldots, n\rangle$. Thus $\THETABODY_l(I) \supseteq \THETABODY_l(I') = [0,1]^n$.
\end{proof}


Let $V_k$ be the subset of $V$ whose elements have at most $k$ entries equal to one. For convenience, we will often identify the elements of $V$, characteristic vectors $\chi_S$ for $S \subseteq \{1,\ldots,n\}$, with their supports, via $S \leftrightarrow \chi_S$. Given $y \in \RR^{V_{2k}}$ we denote the {\em reduced 
moment matrix} of $y$ with respect to $I$ to be the matrix $M_{V_k}(y) \in \RR^{V_k \times V_k}$ defined by 
$$[M_{V_k}(y)]_{X,Y}=
\left\{  
\begin{array}{ll} 
y_{X \cup Y} & \ \ \textrm{ if } X \cup Y \in V, \\
\\
0            & \ \ \textrm{ otherwise. }   
\end{array} 
\right.$$

With these matrices we can finally give a semidefinite description of $\THETABODY_k(I)$.

\begin{proposition}
With $I$ and $V$ as before, $\THETABODY_k(I)$ is the projection onto the coordinates $(y_{e_1}, \cdots, y_{e_n})$ of the set
$$\{y \in \RR^{V_{2k}}\ : \ M_{V_k}(y) \succeq 0 \textrm{ and } y_{0}=1\}.$$
In particular, optimizing to arbitrary fixed precision over $\THETABODY_k(I)$ can be done polynomially in $n$ for fixed $k$.
\end{proposition}

Now we can consider the specific case of the $K_i$-free problem. Here the variety $V \subseteq \RR^{\KK_{i-1}(G)}$ is the set of characteristic vectors of $K_i$-free subsets of $\KK_{i-1}(G)$, $V_k$ is the subset of $V$ of elements of size at most $k$, and $I$ is the vanishing ideal of $V$, described by Lemma \ref{ideal}. Since the $K_i$s in $G$ are the minimal elements not in $V$, by Lemma \ref{ideal} we can write the ideal $I$ as follows.
$$I=\langle x_j^2 - x_j :j \in \KK_{i-1}(G); \prod_{j \subseteq K}x_j: K \in \KK_i(G)\rangle.$$

For example, let $G$ be a triangle, with edges A, B, C, and consider the triangle free problem on $G$. Then the ideal is
$$I = \langle x_A^2 - x_A, x_B^2 - x_B,  x_C^2 - x_C, x_Ax_Bx_C \rangle,$$
and the variety $V$ is as follows.
$$V = \{\emptyset,\{A\},\{B\},\{C\},\{A,B\},\{A,C\},\{B,C\}\} \equiv \{0,1,2,3,4,5,6\}.$$ 
Note that here, we again use our identification of sets with their characteristic vectors. To avoid writing, e.g., $y_{\{A,C\}}$ or even $y_{\chi_{\{A,C\}}}$, we label the elements of $V$ by numbers as above.
Then the moment matrix $M_{V_2}(y)$ is as follows:
$$M_{V_2}(y) = 
\left[
\begin{array}{ccccccc}
y_0 & y_1 & y_2 & y_3 & y_4 & y_5 & y_6 \\
y_1 & y_1 & y_4 & y_5 & y_4 & y_5 & 0 \\
y_2 & y_4 & y_2 & y_6 & y_4 & 0 & y_6 \\
y_3 & y_5 & y_6 & y_3 & 0 & y_5 & y_6 \\
y_4 & y_4 & y_4 & 0 & y_4 &0 & 0 \\
y_5& y_5 & 0 & y_5 & 0 & y_5 & 0 \\
y_6 & 0 & y_6 & y_6 & 0 & 0 & y_6 \\
\end{array}
\right]$$
Projecting the set $\{y: y_0 = 1, M_{V_2}(y) \succeq 0 \}$ onto $(y_1,y_2,y_3)$ gives $\THETABODY_2(I)$ for this graph.

\section{Polynomial-time algorithm}
A graph $H$ is a $K_i$-$p$-hole if $H$ contains $p$ copies of $K_i$ as subgraphs, $G_1, \ldots, G_p$, and $G_j$ and $G_l$ share a common $K_{i-1}$ if and only if $j-l = \pm 1 \mod p$; see Figure \ref{ki-p-hole}. Theorem 3.5 in \cite{conforti} establishes that for $i \ge 3$ and odd $p$, the inequality $\sum_{\KK_{i-1}(H)} x_j \le (\frac{p-1}{2})(2i-3)+i-2$ defines a facet of $P_i(G)$ for each induced $K_i$-$p$-hole $H$ of $G$.
We will show that the facets corresponding to induced $K_i$-$p$-holes are valid on $\THETABODY_{\lceil i/2 \rceil}(I)$, which can be optimized over in polynomial time, addressing an open question in Conforti, Corneil and Mahjoub \cite{conforti}. Note that in this section, the ideal $I$ always refers to the $K_i$ free problem, and the associated graph $G$ will be clear from context.

\begin{figure}[htd]
	\centering
	\includegraphics[width=.3\textwidth]{k3_12_hole_1.png}
	\includegraphics[width=.25\textwidth]{k3_12_hole_4.png}
	\includegraphics[width=.3\textwidth]{k3_12_hole_3.png}
	\caption{Three non-isomorphic $K_3$-12-holes.}
	\label{ki-p-hole}
\end{figure}

The first lemma is an auxiliary result that a class of functions are sums of squares. For an ideal $I$, a function $f$ is said to be {\em idempotent} mod $I$ if $f^2 \equiv f \mod I$. Since an idempotent is visibly a square, we can use it as a summand in our sum of squares. In practice, idempotents end up being very useful in sums of squares.

\begin{lemma}\label{ki}\
Suppose 
$A \subseteq B \subseteq \KK_{i-1}(K_i)$. Denote the variables in $\KK_{i-1}(K_i)$ by $\{x_k:1 \le k \le i\}$.
Then $f(x) = |B \setminus A| - x^A+x^B- \sum_{k\in B\setminus A} x_k $ is $|B|$-sos mod $I$.
\end{lemma}
\begin{proof}
Let $A = A_1 \subset A_2 \ldots \subset A_m = B$ be a maximal chain, where $A_k \cup \{x_k\} = A_{k+1}$, for $k=1,\ldots,m-1$. Check that $g_k(x) = 1-x_k-x^{A_k} + x^{A_{k+1}}$ is idempotent mod $I$. Adding them
up we get that $f(x) = \sum_{k=1}^{m-1} g_k(x)$. Since each summand has degree at most $|B|$ the assertion holds.
\end{proof}

The stable set polytope $\textup{STAB}(G)$ has a fractional relaxation $\textup{FRAC}(G)$, given by imposing nonnegativities $x_i \ge 0$, and inequalities $x_i + x_j \le 1$ for each edge $(i,j)$ of $G$. Similarly, we can define a fractional $K_i$ free polytope $\textup{FRAC}_i(G)$ by imposing nonnegativities, and the inequalities $\sum_{k \in \KK_{i-1}(H)}x_k \le i-1$ for each $H \in \KK_i(G)$. The following corollary shows that these inequalities are $\lceil i/2 \rceil$-sos, and therefore that the relaxation $\THETABODY_{\lceil i/2 \rceil}(I) \subseteq \textup{FRAC}_i(G)$. This is parallel to the result that the Lov\'{a}sz theta body lies inside $\textup{FRAC}(G)$.

\begin{corollary} \label{frac}
The inequality $\sum_{k \in \KK_{i-1}(H)}x_k \le i-1$ is valid on $\THETABODY_{\lceil i/2 \rceil}(I)$ for every $H \in \KK_i(G)$.
\end{corollary}
\begin{proof}
Let $J$ be a subset of $\KK_{i-1}(H)$ of size $\lceil i/2 \rceil$. Applying Lemma \ref{ki} with 
$A=\emptyset$ and $B=J$ we see that 
 $$ f(x) =  |J|-1 + x^J - \sum_{l \in J} x_l $$ is $|J|$-sos. Similarly 
$$g(x)= |J^c|-1 + x^{J^c} - \sum_{l \in J^c} x_l $$ is $|J^c|$-sos. Finally observe that 
$h(x) = 1 - x^J - x^{J^c}$ is idempotent. Since these polynomials are all $\lceil i/2 \rceil$-sos, it remains to observe that their sum,
$$f(x)+g(x)+h(x)=i-1-\sum_{k \in \KK_{i-1}(H)}x_k,$$
is also $\lceil i/2 \rceil$-sos.
\end{proof}

Now we are ready to prove that the $K_i$-$p$-hole inequalities are valid on $\THETABODY_{\lceil i/2 \rceil}(I)$. Recall that if $H$ is a $K_i$-$p$-hole, we write $G_1,\ldots,G_p$ for the $K_i$s in $H$, with adjacent $K_i$ sharing a common $K_{i-1}$. If $G$ has an induced $K_i$-$p$-hole $H$, then the inequality
$$k(2i-3)+i-2 - \sum_{i \in H} x_i \geq 0$$
defines a facet of $P_i(G)$ for $i \ge 3$; see \cite{conforti}.

\begin{lemma}\label{kiphole}
The $K_i$-$p$-hole inequalities are $\lceil i/2 \rceil$-sos for $p$ odd.
\end{lemma}
\begin{proof}

Let $p = 2k+1$. For each $l=1,\ldots,2k+1$, there is exactly one $K_{i-1}$ common to $G_l$ and $G_{l-1}$ (taking indices mod $2k+1$). Denote this variable by $x_l$. Now fix $l$. Let the variables $\{y_k\}$ correspond to the $K_{i-1}$ contained in only $G_l$. Then the variables corresponding to $\KK_{i-1}(G_l)$ are $\{x_l,x_{l+1},y_1,\ldots,y_{i-2}\}$. We will show that $p_l(x,y) = i-2 - \sum y_k - x_lx_{l+1}$ is $\lceil i/2 \rceil$-sos.

Let $J_1 = \{1,\ldots,\lceil i/2 \rceil - 2\}$ and $J_2 = \{\lceil i/2 \rceil - 1, \ldots,i-2\}$. Applying Lemma \ref{ki}, we see that the following two functions are $\lceil i/2 \rceil$-sos. First apply the lemma with $A = \{x_l,x_{l+1}\}$ and $B = \{y_j: j \in J_1\} \cup \{x_l,x_{l+1}\}$:
$$f(x,y)=|J_1| - x_lx_{l+1} + x_lx_{l+1}y^{J_1}- \sum_{j \in J_1} y_j .$$
Second, take $A = \emptyset$ and $B = J_2$:
$$g(x,y)=|J_2| - 1  + y^{J_2}- \sum_{j \in J_2} y_j .$$
Finally, observe that the following is idempotent:
$$h(x,y) = 1-x_lx_{l+1}y^{J_1} - y^{J_2}.$$
Adding these up we get that $p_l(x,y) = f(x,y)+g(x,y)+h(x,y)$ is $\lceil i/2 \rceil$-sos. Now with $p(x,y) = \sum_{l=1}^{2k+1} p_l(x,y)$, we have that $p$ is $\lceil i/2 \rceil$-sos:
$$p(x,y) = (2k+1)(i-2) - \sum_{l=1}^{2k+1}\sum_{y_k \subseteq G_l}y_k - \sum_{l=1}^{2k+1}x_lx_{l+1},$$
where the sum $\sum y_k$ is over all $K_{i-1}$ contained in a unique $K_i$. It remains to show that $k-\sum x_l + \sum x_lx_{l+1}$ is $\lceil i/2 \rceil$-sos. Observe that this is attained by adding the following two quantities, each of which is a sum of idempotents.
$$ \sum_{l=1}^k \left(1 - x_{2l-1} - x_{2l} - x_{2l+1} + x_{2l-1}x_{2l} + x_{2l-1}x_{2l+1} + x_{2l}x_{2l+1}\right)$$
$$ \sum_{l=2}^k (x_{2l-1} - x_{2l-1}x_{1} - x_{2l-1}x_{2l+1} + x_{2l+1}x_{1})$$
\end{proof}

In section 3.3 of \cite{conforti}, Conforti, Corneil, and Mahjoub show that polynomial time separation oracles exist for each of the following facets of $P_i(G)$.
\begin{enumerate}
\item Nonnegativities $0 \le x_k \le 1$,
\item $K_i$ (clique) inequalities $\sum_{k \in K_i}x_k \le i-1$,
\item Odd wheels of order $i-1$.
\end{enumerate}
Define the polytope $Q(G)$ as the intersection of these facets, and define $Q'(G)$ by replacing (3) by 
\begin{enumerate}[label=(\arabic*$'$),ref=(\arabic*$'$),start=3]
\item $K_i$-$p$-hole facets,
\end{enumerate}
a superclass of (3).

The result from Conforti, Corneil, and Mahjoub allows us to optimize over $Q(G)$ in polynomial time. However, it is open in \cite{conforti} whether there is also a polynomial time separation oracle for the class ($3'$), or more generally, whether we can optimize efficiently over $Q'(G)$. 

The results in this section show that we can optimize in polynomial time over $\THETABODY_{\lceil i/2 \rceil}(I)$, a tighter relaxation than $Q'(G)$. Precisely, we have the following inclusions:
$$P_{i}(G) \subseteq  \THETABODY_{\lceil i/2 \rceil}(I) \subseteq Q'(G) \subseteq Q(G).$$.
There are other families of facets of $P_i(G)$ for which efficient separation oracles are given in \cite{conforti}. We have not treated them here, as they would not yield any new polynomial time results.

\section{Related Problems}
Here we apply two results appearing in the literature to the triangle free problem.

\subsection{A lower bound on theta convergence}
In Section 3, we showed that the earliest possible theta body, $\THETABODY_{\lceil i/2 \rceil}(I_G)$, satisfies several inequalities defining facets of $P_i(G)$. However, in general it can take many steps of the theta hierarchy before a given facet of $P_i(G)$ is valid on $\THETABODY_k(G)$. This is the case even for the triangle free problem. In particular, we will show:
\begin{theorem}
For $k < \frac{n-2}{4}$, $P_3(K_n) \subsetneq \THETABODY_{k}(I_{K_n})$.
\end{theorem}

To prove Theorem 4.1, we will apply a result of Laurent on the {\em cut polytope}. Let $G=(N,E)$ be a graph. A {\em cut} in $G$ arises from a partition of the node set $N$ into two sets $S_1$ and $S_2$, whereupon the associated cut is the set of edges from $S_1$ to $S_2$. Let $C_G \subseteq \{0,1\}^E$ be the collection of characteristic vectors of cuts in $G$. Then $\textup{CUT}(G) = \textup{conv}(C_G)$ is the cut polytope of $G$. Similarly, define $T_G \subseteq \{0,1\}^E$ to be the set of characteristic vectors of triangle free sets in $G$, and as before, $P_3(G) = \textup{conv}(T_G)$. Note that a cut is by definition bipartite; hence, it is triangle free. Therefore $C_G \subseteq T_G$ and $\textup{CUT}(G) \subseteq P_3(G)$. The theta body approach has also been applied to the cut polytope by Gouveia, Laurent, Parrilo, and Thomas \cite{GLPT}. The following lemma shows that inclusion among varieties extends to inclusion of theta bodies.

\begin{lemma}\label{ideal_inclusion}
Let $X \subseteq Y$ be two real varieties, with ideals $I(X)$ and $I(Y)$. Then for any $k$, $\THETABODY_k(I(X)) \subseteq \THETABODY_k(I(Y))$.
\end{lemma}
\begin{proof}
If $X \subseteq Y$, then the reverse inclusion holds for their ideals: $I(Y) \subseteq I(X)$. Any function which is $k$-sos mod $I(Y)$ is then also $k$-sos mod $I(X)$. The result follows from the definition of $\THETABODY_k(I)$.
\end{proof}
In particular, since $C_G \subseteq T_G$, we have $\THETABODY_k(I(C_G)) \subseteq \THETABODY_k(I(T_G))$ for all $k$.

For the complete graph $K_n$, when $n$ is odd, the inequality
\begin{equation} \sum_{e \in E} x_e \le \frac{n^2-1}{4}\end{equation}
defines a facet of both $P_3(K_n)$ and $\textup{CUT}(K_n)$ \cite{moniquestuff}. This inequality does not hold on $\THETABODY_k(I(C_{K_n}))$ for $k < \frac{n-2}{4}$; see example 3.9 in \cite{GLPT}.
\begin{theorem}
For $k < \frac{n-2}{4}$, $\textup{CUT}(K_n) \subsetneq \THETABODY_{k}(I(C_{K_n}))$. In particular, equation (1) is not valid on $\THETABODY_{k}(I(C_{K_n}))$.
\end{theorem}

We can now prove Theorem 4.1.

\begin{proof}
By Theorem 4.3, there is a point $x \in \THETABODY_{k}(I(C_{K_n}))$ violating (1). But by Lemma 4.2, $x \in \THETABODY_{k}(I(T_{K_n}))$. Since (1) is valid on $P_3(K_n)$, $x \notin P_3(K_n)$.
\end{proof}

This implies that the theta body hierarchy does not polynomially capture the $K_n$ inequalities, as the size of the reduced moment matrices associated with the $\lceil\frac{n-2}{4}\rceil$-th theta body is exponential in $n$. It is still an open question, for both $\textup{CUT}(K_n)$ and $P_3(K_n)$, what is the
smallest $k$ so that the $k$-th theta body is exact.

\subsection{An integrality gap for triangle cover}
Let $G$ be a graph. A {\it triangle cover} is a collection of edges in $G$, containing at least one edge from every triangle in $G$. Let $\tau(G)$ be the minimum-size triangle cover in $G$ (in the language of the introduction, the $K_3$ cover problem with unit weights). Let $I$ be the ideal of the triangle cover problem. Define the following semidefinite relaxation:
$$\tau^\dagger(G) = \min \left\{\sum_{e \in E} x_e: x \in \THETABODY_2(I) \right\}.$$
Note that since $C$ is a triangle cover if and only if $E \setminus C$ is a triangle free set, we can restate any statements about theta bodies for the triangle free problem using the change of variables $x \mapsto 1-x$.

We can also define a natural LP relaxation for the triangle cover problem. Let
$$\tau^*(G) = \min \left\{\sum_{e \in E} x_e: x \in [0,1]^E \hbox{ and for all triangles $\Delta$,} \sum_{e \in \Delta} x_e \ge 1\right\}.$$
Krivelevich \cite{krivelevich} proved that $\tau(G) \le 2\tau^*(G)$. We can apply this to prove an integrality gap for $\tau^\dagger(G)$.
\begin{theorem}
For any graph $G$, $\tau^\dagger(G) \ge \frac{\tau(G)}{2}$.
\end{theorem}
\begin{proof}
By Corollary 3.2, $\tau^\dagger(G) \ge \tau^*(G)$, as the inequalities defining $\tau^*(G)$ are valid on the second theta body. Combining this with Krivelevich's inequality gives the result.
\end{proof}

Another way to interpret Krivelevich's result is in terms of a conjecture of Tuza. Define a {\it triangle packing} in a graph $G$ to be a collection of triangles in $G$, no two of which share an edge. Let $v(G)$ be the maximum-size triangle packing in $G$. It is an easy exercise to check that $v(G) \le \tau(G) \le 3v(G)$. However, Tuza conjectured in \cite{tuza} that the stronger inequality $\tau(G) \le 2v(G)$ holds for all graphs $G$. The problem is currently open; see \cite{haxell} for more information. $v(G)$ also has a natural LP relaxation.
$$v^*(G) = \max \left\{\sum_{\Delta \in T} y_\Delta: y \in [0,1]^T \hbox{ and for all edges $e$,} \sum_{e \in \Delta} y_\Delta \le 1\right\}$$
By LP duality, $\tau^*(G) = v^*(G)$. Krivelevich also proved that  $v^*(G) \le 2v(G)$. After applying the duality $\tau^*(G) = v^*(G)$, these become fractional versions of Tuza's conjecture: $\tau(G) \le 2v^*(G)$ and $\tau^*(G) \le 2v(G)$. A natural question to ask is whether, given the SDP relaxation $\tau^\dagger(G)$, whether the ``semidefinite version'' of Tuza's conjecture would hold: $\tau^\dagger(G) \le 2v(G)$.





